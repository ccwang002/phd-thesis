\chapter{Parts of the Dissertation}
\label{chap:dissertation-parts}

This chapter describes the components of a dissertation or thesis.
You may not have to include all components described here, but you must follow the prescribed order for the components you do include.
On \pref{tab:include}, \tref{tab:include} lists the required and optional components in the order that they should appear.
Your manuscript should include three main parts: the front matter, the body, and the back matter.
Each of these parts is described below.

\section{Front Matter}

The front matter includes all material that appears before the beginning of the body of the text.
Number all front matter pages (except the title page and the optional copyright page) with lowercase roman numerals, starting with ii, centered just above the bottom margin.
Each of the following sections should begin on a new page.

\subsection{Title Page}

Format the title page so that it is centered vertically and horizontally on the page with equal amounts of white space from top and bottom margins.
Include a 1 inch margin on all sides.
Use a 12-point regular font.
If you are writing a thesis, substitute the word ``thesis'' wherever the word ``dissertation'' appears in this document.
All master's students should not include their thesis examination committee members on the title page.
The date on the title page should reflect the month and year the degree is to be officially earned, and should be one of the following months: December, May, or August.
Do not include a page number on the title page.
See Appendix for further details.
In most cases your dissertation title should be in ``Title Case'' unless a specific format is required by your discipline.
Be certain to use your own full name (as recorded in \href{https://acadinfo.wustl.edu/}{WebSTAC}).

\begin{table}
	\caption[%
        Required and Optional Thesis Components (NOTE: If you have a multi-lined table label/title,
        then the 2\textsuperscript{nd} and all additional lines should align with the first line, just like this one;
        also, be sure that no words display to the far right hand side where the page numbers for
        your tables display, just as shown in this example.)
    ]{The following items may be included in your dissertation or thesis, in the order in which they are listed.
  Any optional components, if used, \underline{must} be included in the table of contents, unless noted below.}
  \label{tab:include}
  \centering
  \DoubleSpacing

  % Because Times New Roman doesn't have checkmark symbol,
  % Use the checkmark symbol from pifont
  \newcommand{\mycheckmark}{\ding{52}}

  \footnotesize
  \begin{threeparttable}[b]
  \begin{tabular}{@{}ccccc@{}}
  \toprule
    \textbf{Major Part} &
    \makecell[b]{\textbf{Thesis} \\ \textbf{Component}} &
    \textbf{Required} &
    \textbf{Optional} &
    \textbf{Page Numbering} \\
  \midrule

  Front Matter & Title page & \mycheckmark & & counted, not numbered \\
  & Copyright page & & \mycheckmark & neither counted, nor numbered \\
  & Table of Contents & \mycheckmark & & begins on page number \underline{ii} \\
  & List of Figures & & \mycheckmark & [lowercase Roman numerals continue] \\
  & List of Illustrations & & \mycheckmark & [lowercase Roman numerals continue] \\
  & List of Tables & & \mycheckmark & [lowercase Roman numerals continue] \\
  & List of Abbreviations & & \mycheckmark & [lowercase Roman numerals continue] \\
  & Acknowledgments & \mycheckmark & & [lowercase Roman numerals continue] \\
  & Dedication\tnote{*} & & \mycheckmark & [lowercase Roman numerals continue] \\
  & Abstract page & & \mycheckmark & [lowercase Roman numerals continue] \\
  & Preface & & \mycheckmark & [lowercase Roman numerals continue] \\
  Body & Epigraph\tnote{*} & & \mycheckmark & begins on a page numbered \underline{1} \\
  & Chapters & \mycheckmark & & [Arabic numerals begin or continue] \\
  Back Matter & References\tnote{**} & \mycheckmark & & [Arabic numerals continue] \\
  & Appendices & & \mycheckmark & [Arabic numerals continue] \\
  & Curriculum Vitae\tnote{***} & & \mycheckmark & [Arabic numerals continue] \\
  \bottomrule
  \end{tabular}

  \vspace{.5em}
  \begin{tablenotes}
    \footnotesize
    \item[*] Do not include in the table of contents.
    \item[**] There are two options for the placement of references; they can be listed at the end of each chapter, or at the end of the document.
    \item[***] Do not put your Social Security Number, birthdate, or birthplace on your CV.
  \end{tablenotes}
  \end{threeparttable}
\end{table}

\subsection{Copyright Page}

It is always suggested that upon completion of the text, the student add the copyright symbol © with the year and the student's name on one line, on a page following the title page.
Format your copyright page exactly as it is shown in this template.

\vspace{\onelineskip}
\noindent
\textit{Example:}

\centerline{\textcopyright\ 2017, Paige Turner}
\vspace{\onelineskip}

Once you create a work, it is automatically protected by U.S.\@ copyright law with you as the author.
You do not need to register the copyright with the U.S.\@ Copyright Office, though doing so provides certain advantages.
More information about copyright registration can be found at \href{http://libguides.wustl.edu/copyright/registration}{http://libguides.wustl.edu/copyright/registration}.

\subsection{Table of Contents}

The words ``Table of Contents'' must appear in chapter title style at the top of the page.
It must include the page numbers of all front and back matter elements, unless otherwise specified.
The table of contents must include the page numbers of all chapters and sections of your dissertation or thesis.
In addition, it may include the page numbers of all subsections.
Chapter titles may be typed in plain or bold font.
All titles and headings must be followed by a page number.

Make certain that any long titles align nicely with the body of text.
Multi-lined chapter titles or section titles should break at a logical point and align in a manner allowing the titles to be read clearly, without confusion.
Sometimes this will mean forcing a line break at a logical point and relies on your own good judgment.

\subsection{List of Figures}

If one or more figures are used in the document, there must be a list of all figures.
The list should be spaced at 1.15.
Begin each listing on a new line.
Format the list of figures the same way the table of contents is formatted, but put the words ``List of Figures'' in the heading.

\subsection{List of Tables}

If one or more tables are used in the document, there must be a list of all tables.
The list should be spaced at 1.15.
Begin each listing on a new line.
Format the list of tables the same way the table of contents is formatted, but put the words ``List of Tables'' in the heading.

\subsection{List of Abbreviations}

Include a list of abbreviations only if you use abbreviations that are not common in your field.
Arrange the list alphabetically.
Type the words ``List of Abbreviations'' in chapter title style at the top of your list.

\subsection{Acknowledgments}

An acknowledgments section must be included.
Use it to thank those who supported your research through contributions of time, money, or other resources.
Some grants require an acknowledgment.
Type the word ``Acknowledgments'' in chapter title style at the top of your page.
If the acknowledgments fill more than one page, put the heading only on the first page.

\subsection{Dedication}

The dedication page is optional.
If you decide to include a separate dedication page, make it short and center it on the page, both horizontally and vertically.
Do not include it in your table of contents.
See \pref{thesisdedication} for more detailed information.

\subsection{Abstract Page}

An abstract page is optional in the dissertation or thesis, but will be required when you submit your manuscript electronically.
Format the abstract page precisely as shown in the front matter of this document.
See Appendix~\ref{app:degree-program} for further details.

\subsection{Preface}

A preface is optional.
If you include a preface, use it to explain the motivation behind your work.
Format the preface the same way the acknowledgments section is formatted, but use the word ``Preface'' in the heading.

\section{Body of the Dissertation or Thesis}

The body of the dissertation or thesis should be divided into chapters, sections, and subsections as required by your discipline, and should be numbered as in this template.
Divisions smaller than subsections may be used, but they should not be labeled with numbers (see \ref{subsec:subsection-headings-numbering} for more information).

\section{Back Matter}

The back matter includes all material that appears after the body of the text.

\subsection{References/Bibliography/Works Cited}

There are two options for the placement of references; they can be listed at the end of each chapter or at the end of the document.
What you call this section and how you format it should follow the usual convention of your discipline and be acceptable to your committee.
Depending on how you title and where you place this section, type the appropriate words in either the section heading or chapter title format at the top of a new page.
Single space your citations and skip a line between each one.
Regardless of where you choose to place your references, they must be listed in the table of contents.
If placed at the end of your document, this section should follow the conclusion of the text.

\subsection{Appendices}

Appendices may be used for including reference material that is too lengthy or inappropriate for the dissertation or thesis body.
If one appendix is included, an appendix title is optional.
If more than one appendix is included, each one should be titled and lettered.
In general, appendices should be formatted like chapters.
However, they may be single-spaced and/or include photocopied or scanned materials.
If these are used, you must add page numbers at the bottom, putting those page numbers in square brackets to indicate that they are not part of the original document.

\subsection{Curriculum Vitae}

Including a Curriculum Vitae (CV) with your dissertation or thesis is optional.
If you choose to include your CV, it should include your name, the month \& year you will be earning your degree, and relevant academic and professional achievements.
It may also include your publications and professional society memberships.
If included, your vita should be the last page(s) of your document.
Note that personally identifiable information such as birth date, place of birth, and social security number should \underline{NOT} be included.
