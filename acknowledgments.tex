\thesisacknowledgments
% Structure:
% Thank Li, thesis committee, lab members
% CPTAC collaborators and other collaborators
% patients data
% friends
% family
% Clarice

Cancer genomics has been my dream research and I was so thrilled when I started my rotation in Ding Lab, the lab that I have been reading about in the publications.
Shortly into my rotation, the lab moved into the new McKinley Building (before it's called Couch Biomedical Research Building).
% epiphany?
When I joined the first lab meeting there and was shown to the lab cubicles next to the large and beautiful windows, I suddenly had this feeling that \textit{this is it; I am part of the team}.
Today, we are in an exciting time of cancer genomics, restlessly pushing the boundaries of the field to uncover new understanding of cancer.
New projects are cool as ever, while I am now wrapping up mine and passing my desk to the new student.
The lab space shows a bit of wear and tear if I squinted hard.
But when it comes to research, I can still can feel the same spark of joy I had the first day I joined the lab.
Such feeling will last forever.
I will greatly miss this wonderful place and all the people.

I would like to first thank my PhD advisor, Dr. Li Ding, for her profound vision and great management skills.
Li always has the right insights into new questions and challenges, guiding me to stay at the right track all along my graduate study.
Li shows me how to lead a large-scale collaboration, balancing between steering the projects at a high level and navigating through the intricacies at the right time.
Her perseverance and passion motivates me to be a better scientist.
I am grateful for her continued support to persue what I love in life.

I am indebted to my thesis committee for the constant support and guidance over my research and career development.
I want to thank Dr. Josh Rubin, for his guidance to develop my thesis especially during a challenging time of my research and his dazzling charisma as a scientist that always channels his passion of glioma research to me.
I am grateful for Dr. Steve Oh's wisdom and humor in our early on imaging meetings that unfortunately did not become part of my thesis, and I truly enjoy his leadership and our discussions about everything.
I would like to thank Dr. Milan Chheda, for his invaluable insights and feedback to my glioblastoma projects as a collobarator. Our virtual meetings would not have been fun and exciting without him. I am also very grateful for Dr. John Edwards's help on my thesis direction and overall guidance that make my thesis complete.

% Thank Li


Steven Foltz, Michael Wendl, Song Cao, Matthew Wyczalkowski, Sohini Sengupta, Yize Li, Adam Scott, Clara Oh, Yanyan Zhao, Alla Karpova, Qingsong Gao, Preet Lal, R. Jay Mashl, Sunantha Sethuraman, Matthew Bailey, Dan Cui, Kuan-Lin Huang, Wen-Wei Liang, Ruiyang Liu, Liang-Bo Wang, Yige Wu, Chris Yoon, Terrence Tsou, Wen-Wei Liao, Venkata Yellapantula, Kai Ye, Jie Ning, Beifang Niu, Jiayin Wang, Mingchao Xie, Fernanda Martins Rodrigues, Yige Wu, Lijun Yao, Dawn King, Mo Huang, Charles Lu, Amila Weerasinghe, Erik Storrs and Hua Sun.

An acknowledgments page must be included in your final dissertation or thesis.  If you wish to
include a special dedication you can either use it to close the acknowledgments page or place it on
the page that immediately follows.  The acknowledgments page should be listed in the table of
contents.  Place it after the final list used in the document, and before any dedication, abstract,
or epigraph that is included.

It is appropriate to acknowledge sources of academic and financial support; some fellowships and
grants require acknowledgment.

We offer special thanks to the Washington University School of Engineering for allowing us to use
their dissertation and thesis template as a starting point for the development of this document.

\null\hfill \thesisauthor

\noindent
\textit{Washington University in St.\@ Louis}\\
\textit{December 2021}
