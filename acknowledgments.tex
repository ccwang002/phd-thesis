\thesisacknowledgments
% Structure:
% Thank Li, thesis committee, lab members
% CPTAC collaborators and other collaborators
% patients data
% friends
% family
% Clarice

Cancer genomics has been my dream research and I was so thrilled when I started my rotation in Ding Lab, the lab that I have been reading about in the publications.
Shortly into my rotation, the lab moved into the new McKinley Building.
When I joined the first lab meeting there and was shown to the lab cubicles next to the large and beautiful windows, I suddenly felt that \textit{this is it. I am part of the team}.
Today, we are in an exciting time of cancer genomics and our lab is relentlessly pushing the boundaries of the field.
New projects are as cool as ever, while I am now wrapping up mine and passing my desk to the new student.
But when it comes to research, I can still can feel the same spark of joy I had the first day I joined the lab.
I will greatly miss this wonderful place and all the people.

I would like to first thank my PhD advisor, Dr. Li Ding, for her profound vision and great management skills.
Li always has the right insights into new questions and challenges, guiding me to stay on the right track along my graduate study.
Li showed me how to lead a large-scale collaboration, balancing between steering the projects at a high level and navigating through the intricacies of the project at the right times.
Her perseverance and passion motivate me to be a better scientist.
I am grateful for her continued support to pursue what I love in life.

I am indebted to my thesis committee for the constant support and guidance over my research and career development.
I want to thank Dr. Josh Rubin, for his guidance in developing my thesis especially during the most challenging time of my research.
His charisma as a scientist with his passion for glioma research always leaves me feeling excited and motivated.
I am grateful for Dr. Steve Oh's wisdom and humor in our early on imaging meetings that unfortunately did not become part of my thesis, and I truly enjoy his leadership and our discussions about everything.
I would like to thank Dr. Milan Chheda, for his invaluable insights and feedback to my glioblastoma projects as a collaborator.
Our virtual meetings would not have been fun and exciting without him.
I am also very grateful for Dr. John Edwards's help on my thesis direction and overall guidance that made my thesis complete.

A huge shoutout to my wonderful Ding Lab members.
We had a lot of fun and I cherish the moments and achievements we made together as a team.
I am privileged to work in such a happening environment and I learned a lot from everyone in the lab.

Big thanks to all my collaborators at TCGA, CPTAC, and HTAN for sharing their wisdom and experience with me.
It was my pleasure to work with all the experts from different fields on a shared purpose.
I would like to especially thank everyone in the CPTAC glioblastoma working group for their belief in my ability to lead.
Their professionalism is what I constantly aspire to live up to.
I am so proud to be part of the team.
I hope I will have the opportunity to meet everyone on the team in person someday.

I am truly grateful for all the patients and their family around the globe who participated in the studies.
They are the heroes who made all the scientific research possible.

I want to thank McDonnell International Scholars Academy and Taiwan MOE--WUSTL Fellowship for their financial support for my graduate study.

Outside of my lab and campus, my life in St. Louis was colorful and full of memories because of my friends and their suppport.
I want to thank them to help me settle in this city and eventually enjoy the midwestern vibe.
Also, I want to thank my friends in Taiwan for they jokes always lighten my heart and I am so glad to see everyone went on our own grant journey to chase our dream.
Friends from the open source communities always have a special place in my heart.
They taught me how to collaborate at a large and remote scale, how to be a practical problem solver, and how to continuously pursue my dream with a healthy work-life balance even before the start of my PhD.
Those lessons I learned from the OSS projects truly complemented my PhD training and greatly benefited my research.

I want to thank my parents who always have my back and let me freely explore my interest.
They are my role models for how to be kind to others and how to approach life and challenges with the right attitude.
Their unconditional love and support from the opposite side of Earth warms my heart dearly.

Finally, I want to thank my wife, Clarice, who constantly inspired me what great science should be and supported me through the darkest and the most difficult times in my PhD study.
Everyday I am amazed and motivated by her passion towards science, her wisdom and knowledge about resarch and everything in life, and her creativity to make things so fun and all the obstacles and failures so easy to overcome.
She is my muse.
She is my life.

\null\hfill \thesisauthor

\noindent
\textit{Washington University in St.\@ Louis}\\
\textit{December 2021}
