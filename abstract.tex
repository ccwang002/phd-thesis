\begin{abstract}
Precision oncology aims to develop personalized therapeutic options based on molecular profiling of an individual’s tumor, which requires accurate interpretation of mutations and dissection of tumor microenvironment. Large-scale sequencing studies have provided us valuable multi-omics datasets across thousands of samples and insights into personalized treatment. However, comparison across studies requires sophisticated data harmonization to remove sequencing artifacts and batch effects. Additionally, these studies lack insight into the protein interactions and post-translational modifications (PTMs), which are the backbone of cell signaling and commonly dysregulated in cancer. My thesis work contributes to precision oncology with two directions: expanding the computational toolbox for multi-omics analysis and building insights into disease biology and treatment of glioblastoma, a deadly cancer with little personalized treatment. First, we profiled the genomic data harmonization pipelines on Genomic Data Commons (GDC) to ensure they can be confidently used on multiple large-scale studies. We showed that the harmonization yields concordant scientific findings to the original studies, and the technical artifacts introduced by the pipelines are removed or documented. With the advent of multiplexed mass spectrometry, Clinical Proteomic Tumor Analysis Consortium (CPTAC) provides genomic and proteomic characterization of treatment-naïve tumors, offering insights into the mutational impact on PTM and signaling transduction. To help the research community access to the proteogenomic datasets, we developed a web portal, PTMcosmos, to integrate the experimentally collected PTM sites from CPTAC to their known function from the literature. We analyzed the PTM abundance change in cancer driver genes and demonstrated their role in oncogenic pathways and their association with different tumor subtypes and clinical outcome. We also reported the mutational impact of cancer driver genes on spatially close-by PTM sites using protein structures. The second part of my thesis investigated the proteogenomic characterization of glioblastoma (GBM), a deadly brain tumor without personalized treatment. We used multi-omics clustering to identify molecular subtypes of GBM tumors with distinct clinical outcome and immune composition differences. Phosphoproteomic analysis identified targetable principal switches mediating RAS pathway activation of tumors with different genetic alterations. We also dissected the tumor microenvironment using single cell transcriptomics and identified the expression signature of tumor associated macrophages and histopathology imaging features associated with different immune subtypes using deep learning. Overall, our work points to new additional therapeutic avenues to stratify patients for appropriate and effective treatment. Together, my thesis contributes to a better toolbox for cancer proteogenomic characterization and insights into GBM disease biology that can inspire novel therapeutic avenues.
\end{abstract}
