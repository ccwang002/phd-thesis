\begin{abstract}
The goal of precision oncology is to develop personalized therapeutic options based on molecular profiling of an individual's tumor. To achieve this goal, we first need accurate interpretation of mutations and dissection of the tumor microenvironment. Large-scale sequencing studies have provided us with valuable multi-omics datasets across thousands of samples and insights into personalized treatment. However, comparison across studies requires sophisticated data harmonization to remove sequencing artifacts and batch effects. Additionally, these studies lack insight into the protein interactions and post-translational modifications (PTMs), which underlie cell signaling pathways that are frequently dysregulated in cancer.
My thesis work contributes to precision oncology with two directions: expanding the computational toolbox for multi-omics analysis and building insights into disease biology and treatment of glioblastoma (GBM), a deadly cancer with little personalized treatment.
First, we profiled the genomic data harmonization pipelines on Genomic Data Commons (GDC) to ensure they can be confidently used on multiple large-scale studies and identified the technical artifacts introduced by different pipelines. Additionally, with the advent of multiplexed mass spectrometry, Clinical Proteomic Tumor Analysis Consortium (CPTAC) combines genomic and proteomic characterization of tumors across cancer types, offering insights into the mutational impact on PTM and signaling transduction.
To help the research community access the proteogenomic datasets, we developed a web portal, PTMcosmos, to link the experimentally collected PTM sites from CPTAC to their known functions from the literature. We analyzed the change in PTM abundance in cancer driver genes and demonstrated their association with oncogenic pathways in different tumor subtypes with various clinical outcomes. We also reported the mutational impact of cancer driver genes on spatially close-by PTM sites using 3D protein structures.
In the second part of my thesis, we performed proteogenomic characterization of GBM, a deadly brain tumor without personalized treatment. We used multi-omics clustering to identify molecular subtypes of GBM tumors with distinct clinical outcomes and immune composition differences. Phosphoproteomic analyses identified the druggable principal switches of RAS pathway activation in tumors with distinct upstream genetic alterations. We also dissected the tumor microenvironment using single cell transcriptomics and identified the expression signature of tumor associated macrophages and histopathology imaging features associated with different immune subtypes using deep learning. Overall, our work points to new additional therapeutic avenues to stratify patients for appropriate and effective treatment.
Together, my thesis contributes to a better toolbox for cancer proteogenomic characterization and insights into GBM disease biology that will lead to novel therapeutic avenues.
\end{abstract}
